\documentclass{article}
\usepackage{amsmath}
\usepackage{braket}
\begin{document}


\section*{Deutsch Algorithm: Full Derivation}

We begin with two qubits initialized as
\[
\ket{\psi_0} = \ket{0}\ket{1}.
\]

\subsection*{1. Apply Hadamard Gates}

Applying the Hadamard to each qubit gives
\[
H\ket{0} = \frac{\ket{0}+\ket{1}}{\sqrt{2}}, 
\qquad
H\ket{1} = \frac{\ket{0}-\ket{1}}{\sqrt{2}}.
\]

Thus
\[
\ket{\psi_1}
= (H\ket{0}) \otimes (H\ket{1})
= \frac{1}{2}(\ket{0}+\ket{1})(\ket{0}-\ket{1}).
\]

Expanding:
\[
\ket{\psi_1}
= \frac{1}{2}\big( \ket{00} - \ket{01} + \ket{10} - \ket{11} \big).
\]

\subsection*{2. Apply the Oracle $U_f$}

The oracle acts as follows:
\[
U_f \ket{x,y} = \ket{x,\, y \oplus f(x)}.
\]

Because the second qubit is in the state
\(
\frac{1}{\sqrt{2}}(\ket{0}-\ket{1}),
\)
the oracle adds a phase $(-1)^{f(x)}$ depending on $x$.  
Thus the state becomes
\[
\ket{\psi_2}
= \frac{1}{\sqrt{2}}\left( (-1)^{f(0)}\ket{0} + (-1)^{f(1)}\ket{1} \right)
\otimes \frac{1}{\sqrt{2}}(\ket{0}-\ket{1}).
\]

We only care about the \emph{top} qubit for the final measurement, so define
\[
\ket{\psi_{\text{top}}}
= \frac{1}{\sqrt{2}}\left( (-1)^{f(0)}\ket{0} + (-1)^{f(1)}\ket{1} \right).
\]

\subsection*{3. Apply the Final Hadamard}

Applying $H$ to the top qubit:

\[
H\ket{0} = \frac{\ket{0}+\ket{1}}{\sqrt{2}}, \qquad
H\ket{1} = \frac{\ket{0}-\ket{1}}{\sqrt{2}}.
\]

Thus
\[
H\ket{\psi_{\text{top}}}
= \frac{1}{\sqrt{2}}
\left[
(-1)^{f(0)} H\ket{0}
+
(-1)^{f(1)} H\ket{1}
\right].
\]

Substituting the Hadamard results:
\[
H\ket{\psi_{\text{top}}}
=
\frac{1}{2}\Big[
(-1)^{f(0)}(\ket{0}+\ket{1})
+
(-1)^{f(1)}(\ket{0}-\ket{1})
\Big].
\]

Group the $\ket{0}$ and $\ket{1}$ terms:
\[
H\ket{\psi_{\text{top}}}
=
\frac{1}{2}\left[
\left((-1)^{f(0)} + (-1)^{f(1)}\right)\ket{0}
+
\left((-1)^{f(0)} - (-1)^{f(1)}\right)\ket{1}
\right].
\]

\subsection*{4. Evaluate All 4 Possible Functions}

There are four Boolean functions $f:\{0,1\}\to\{0,1\}$.

\begin{align*}
f(0)=0,\ f(1)=0 \quad &\Rightarrow\quad H\ket{\psi_{\text{top}}} = \ket{0}, \\
f(0)=1,\ f(1)=1 \quad &\Rightarrow\quad H\ket{\psi_{\text{top}}} = \ket{0}, \\
f(0)=0,\ f(1)=1 \quad &\Rightarrow\quad H\ket{\psi_{\text{top}}} = \ket{1}, \\
f(0)=1,\ f(1)=0 \quad &\Rightarrow\quad H\ket{\psi_{\text{top}}} = \ket{1}.
\end{align*}

\subsection*{5. Final Measurement}

\[
\boxed{
\begin{cases}
\ket{0}, & \text{if $f$ is constant}, \\
\ket{1}, & \text{if $f$ is balanced}.
\end{cases}
}
\]

Thus the Deutsch algorithm determines whether $f$ is constant or balanced using a single oracle query.

\end{document}
