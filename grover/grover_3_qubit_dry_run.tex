\documentclass{article}
\usepackage{amsmath}
\usepackage{amssymb}
\usepackage{geometry}
\usepackage{booktabs}

\geometry{a4paper, margin=1in}

\title{Grover's Algorithm: 3-Qubit Step-by-Step Derivation}
\author{}
\date{}

\begin{document}

\maketitle

\section{Setup}

For a 3-qubit system, we have $N = 2^3 = 8$ items.
\begin{itemize}
    \item \textbf{Target state ($|w\rangle$):} We assume the target is $|111\rangle$.
    \item \textbf{Initial Amplitude ($a$):} The system starts in a uniform superposition, so every state has an amplitude of $\frac{1}{\sqrt{8}} \approx 0.354$.
    \item \textbf{Initial State ($|\psi_0\rangle$):}
    \begin{equation}
        |\psi_0\rangle = \sum_{x \in \{0,1\}^3} \frac{1}{\sqrt{8}} |x\rangle = \left[ \frac{1}{\sqrt{8}}, \frac{1}{\sqrt{8}}, \dots, \frac{1}{\sqrt{8}} \right]
    \end{equation}
    \item \textbf{Initial Probability of measuring $|w\rangle$:}
    \begin{equation}
        P(w) = \left| \frac{1}{\sqrt{8}} \right|^2 = \frac{1}{8} = 12.5\%
    \end{equation}
\end{itemize}

\section{Run 1: First Iteration}

\subsection{Step 1: The Oracle}
The Oracle flips the sign of the target amplitude ($|w\rangle$).
\begin{equation}
    |\psi_{\text{oracle}}\rangle = \left[ \frac{1}{\sqrt{8}}, \dots, \mathbf{-\frac{1}{\sqrt{8}}}, \dots \right]
\end{equation}
\begin{itemize}
    \item Non-winners ($x \neq w$): $\frac{1}{\sqrt{8}} \approx 0.354$
    \item Winner ($w$): $-\frac{1}{\sqrt{8}} \approx -0.354$
\end{itemize}

\subsection{Step 2: The Diffuser (Inversion about the Mean)}
We calculate the mean ($\mu$) of all amplitudes in $|\psi_{\text{oracle}}\rangle$:
\begin{equation}
    \mu = \frac{1}{N} \sum_i \alpha_i = \frac{1}{8} \left( 7 \times \frac{1}{\sqrt{8}} + 1 \times \frac{-1}{\sqrt{8}} \right) = \frac{1}{8} \left( \frac{6}{\sqrt{8}} \right) = \frac{3}{4\sqrt{8}} \approx 0.265
\end{equation}

Now, we apply the operation $2\mu - \alpha_i$ to each amplitude:

\begin{itemize}
    \item \textbf{For Non-winners ($x$):}
    \begin{equation}
        2(0.265) - 0.354 = 0.530 - 0.354 = \mathbf{0.177} \approx \frac{1}{4\sqrt{2}}
    \end{equation}
    \item \textbf{For Winner ($w$):}
    \begin{equation}
        2(0.265) - (-0.354) = 0.530 + 0.354 = \mathbf{0.884} \approx \frac{5}{4\sqrt{2}}
    \end{equation}
\end{itemize}

\textbf{State after Run 1 ($|\psi_1\rangle$):}
\begin{equation}
    |\psi_1\rangle = [ 0.177, \dots, \mathbf{0.884}, \dots ]
\end{equation}
The probability of measuring the target is now $|0.884|^2 = \mathbf{78.1\%}$.

\section{Run 2: Second Iteration}

\subsection{Step 1: The Oracle}
We flip the sign of the winner $|w\rangle$ again.
\begin{itemize}
    \item Non-winners: $0.177$
    \item Winner: $\mathbf{-0.884}$
\end{itemize}

\subsection{Step 2: The Diffuser}
Calculate the new mean ($\mu$):
\begin{equation}
    \mu = \frac{1}{8} \left( 7 \times 0.177 + 1 \times (-0.884) \right) = \frac{1}{8} (1.239 - 0.884) = \frac{0.355}{8} \approx \mathbf{0.044}
\end{equation}
\textit{Note: The mean drops significantly because the large negative amplitude of the winner pulls it down.}

Apply $2\mu - \alpha_i$:

\begin{itemize}
    \item \textbf{For Non-winners ($x$):}
    \begin{equation}
        2(0.044) - 0.177 = 0.088 - 0.177 = \mathbf{-0.088}
    \end{equation}
    \item \textbf{For Winner ($w$):}
    \begin{equation}
        2(0.044) - (-0.884) = 0.088 + 0.884 = \mathbf{0.972}
    \end{equation}
\end{itemize}

\textbf{State after Run 2 ($|\psi_2\rangle$):}
\begin{equation}
    |\psi_2\rangle = [ -0.088, \dots, \mathbf{0.972}, \dots ]
\end{equation}
The probability of measuring the target is now $|0.972|^2 = \mathbf{94.5\%}$.

\section{Conclusion and Summary}

For $N=8$, the optimal number of iterations is 2. If we were to run a 3rd iteration, the state would ``over-rotate,'' and the probability would drop to approximately 33\%.

\begin{table}[h]
    \centering
    \begin{tabular}{cc}
        \toprule
        \textbf{Iteration} & \textbf{Probability of Finding Target ($|w\rangle$)} \\
        \midrule
        0 (Start) & 12.5\% \\
        1 & 78.1\% \\
        \textbf{2 (Optimal)} & \textbf{94.5\%} \\
        3 (Over-rotated) & 33.0\% \\
        \bottomrule
    \end{tabular}
    \caption{Probability improvement per iteration for 3 qubits.}
\end{table}

\end{document}