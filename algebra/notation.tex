\documentclass[a4paper,11pt]{article}

% Packages for math and formatting
\usepackage[utf8]{inputenc}
\usepackage{geometry}
\geometry{margin=1in} % Sets comfortable margins
\usepackage{amsmath}
\usepackage{amssymb}
\usepackage{physics} % ESSENTIAL: Provides \bra{}, \ket{}, \braket{}, \ketbra{}
\usepackage{booktabs} % For nicer looking tables
\usepackage{xcolor}
\usepackage{hyperref}

\title{\textbf{Quantum Algebra: A Visual Guide to Bra-Ket Notation}}
\author{Study Sheet}
\date{}

\begin{document}

\maketitle

\section*{Introduction}
In quantum computing, notation can be tricky because symbols are often dropped for brevity. This guide defines the three major products using visual cues.

% --- SECTION 1: INNER PRODUCT ---
\section{The Inner Product (``Bra-Ket'')}
\begin{itemize}
    \item \textbf{Notation:} $\braket{a}{b}$
    \item \textbf{Visual Cue:} Pointy brackets facing \textbf{inward} like a diamond $\langle \dots \rangle$.
    \item \textbf{Meaning:} Calculates the \textbf{Overlap} or similarity between two states.
    \item \textbf{Result:} A \textbf{Scalar Number} (e.g., $0, 1, 0.5$).
\end{itemize}

\paragraph{Example from Grover's Algorithm:}
In the Grover example, you encountered $\braket{w}{s} = \frac{1}{2}$.
\begin{quote}
    This asks: ``How much of the target $\ket{w}$ is hiding inside the superposition $\ket{s}$?'' The answer is the number $0.5$.
\end{quote}

% --- SECTION 2: OUTER PRODUCT ---
\section{The Outer Product (``Ket-Bra'')}
\begin{itemize}
    \item \textbf{Notation:} $\ketbra{a}{b}$
    \item \textbf{Visual Cue:} Pointy brackets facing \textbf{outward} like two funnels back-to-back $\ket{\dots}\bra{\dots}$.
    \item \textbf{Meaning:} Creates an \textbf{Operator} (a Matrix) that transforms vectors. Read it as ``A machine that accepts state $b$ and transforms it into state $a$''.
    \item \textbf{Result:} A \textbf{Matrix}.
\end{itemize}

\paragraph{Example from Grover's Algorithm:}
The operator $2\ketbra{s}{s}$ is an outer product.
\begin{itemize}
    \item $\bra{s}$ is the ``sensor'': measures how much input is parallel to $\ket{s}$.
    \item $\ket{s}$ is the ``builder'': reconstructs that amount in the direction of $\ket{s}$.
    \item Together, they act as a \textbf{Projector}.
\end{itemize}

% --- SECTION 3: TENSOR PRODUCT ---
\section{The Tensor Product (``Ket-Ket'')}
\begin{itemize}
    \item \textbf{Notation:} $\ket{a} \otimes \ket{b}$ (often written as $\ket{a}\ket{b}$ or $\ket{ab}$).
    \item \textbf{Visual Cue:} Two vertical bars or Kets side-by-side $\ket{\dots}\ket{\dots}$.
    \item \textbf{Meaning:} \textbf{Glues} smaller systems together to make a larger system.
    \item \textbf{Result:} A \textbf{Larger Vector}.
\end{itemize}

\paragraph{Example from Grover's Algorithm:}
The starting state $\ket{0} \otimes \ket{0}$ (or $\ket{00}$) describes ``Qubit 1 is 0 AND Qubit 2 is 0''.

% --- CHEAT SHEET TABLE ---
\section*{The Cheat Sheet: Look at the Brackets}
\begin{center}
\begin{tabular}{@{} l l l l l @{}} % @{} removes outer padding
\toprule
\textbf{Notation} & \textbf{Name} & \textbf{Visual Shape} & \textbf{Result} & \textbf{Meaning} \\ 
\midrule
$\braket{A}{B}$ & Inner Product & $\langle \ \rangle$ (Closed) & Number & ``How similar are A and B?'' \\ 
\addlinespace
$\ketbra{A}{B}$ & Outer Product & $\rangle \ \langle$ (Open) & Matrix & Transformation / Projection \\ 
\addlinespace
$\ket{A}\ket{B}$ & Tensor Product & $| \ \rangle \ | \ \rangle$ (Parallel) & Vector & Combined System \\ 
\bottomrule
\end{tabular}
\end{center}
% 

\newpage

% --- CONCRETE MATH SECTION ---
\section*{A Concrete Math Example}
Let's define standard bit states as vectors:
\[
\ket{0} = \begin{bmatrix} 1 \\ 0 \end{bmatrix}, \quad \ket{1} = \begin{bmatrix} 0 \\ 1 \end{bmatrix}
\] % 

\subsection*{1. Inner Product (Scalar)}
Calculates overlap. Since $\ket{0}$ and $\ket{1}$ are orthogonal:
\[
\braket{0}{1} = \begin{bmatrix} 1 & 0 \end{bmatrix} \begin{bmatrix} 0 \\ 1 \end{bmatrix} = (1)(0) + (0)(1) = \mathbf{0}
\] % [cite: 33]

\subsection*{2. Outer Product (Matrix)}
Transforms $\ket{1}$ into $\ket{0}$:
\[
\ketbra{0}{1} = \begin{bmatrix} 1 \\ 0 \end{bmatrix} \begin{bmatrix} 0 & 1 \end{bmatrix} 
= \begin{bmatrix} 1\cdot0 & 1\cdot1 \\ 0\cdot0 & 0\cdot1 \end{bmatrix} 
= \begin{bmatrix} 0 & \mathbf{1} \\ 0 & 0 \end{bmatrix}
\] % [cite: 34, 35]

\subsection*{3. Tensor Product (Larger Vector)}
Expands the system state space:
\[
\ket{0} \otimes \ket{1} = \begin{bmatrix} 1 \\ 0 \end{bmatrix} \otimes \begin{bmatrix} 0 \\ 1 \end{bmatrix} 
= \begin{bmatrix} 1 \cdot \begin{bmatrix} 0 \\ 1 \end{bmatrix} \\ 0 \cdot \begin{bmatrix} 0 \\ 1 \end{bmatrix} \end{bmatrix} 
= \begin{bmatrix} 0 \\ 1 \\ 0 \\ 0 \end{bmatrix}
\] % [cite: 36, 37]

\end{document}
